\section{Introducción}
En este proyecto vamos a abordar el problema de la colorización automática de imágenes en blanco y negro. Esta tarea consiste en inferir la información cromática que no está presente en la imagen original, esto hace que sea un problema inherentemente ambiguo ya que para una misma imagen puede haber varias colorizaciones que sean creíbles y coherentes. El objetivo no es únicamente generar imágenes visualmente atractivas, sino que estos resultados tambien sean coherentes con la estructura y el contenido semántico de la imagen, evitando inconsistencias y artefactos perceptuales.
Para este proyecto hemos explorado diferentes enfoques, desde \textbf{Autoencoders} y \textbf{Variational Autoencoders}, que aprenden a comprimir y reconstruir imágenes en espacios latentes, hasta \textbf{redes UNet}, que aprovechan las skip conexions para preservar los detalles espaciales durante la reconstrucción.

Por otro lado, las \textbf{Generative Adversarial Networks GAN} han demostrado un gran potencial para generar colorizaciones realistas, gracias a la dinámica competitiva entre generador y discriminador. En colorización, las GAN permiten producir colores más vivos y naturales, aunque a menudo suponen un entrenamiento más inestable. Más recientemente, los \textbf{modelos de difusión} han surgido como una alternativa poderosa, ya que son capaces de generar imágenes de alta calidad mediante un proceso iterativo refinando el ruido hacia imágenes de buena calidad. Estos modelos, al operar en espacios latentes comprimidos por un VAE, han mostrado lograr muy buenos resultados en tareas como la colorización.

Otro aspecto adicional que también ha cobrado relevancia es el \textbf{condicionamiento mediante texto}. La posibilidad de guiar el proceso de colorización con descripciones abre la puerta a aplicaciones creativas y personalizadas, donde el usuario puede especificar los estilos, paletas cromáticas o incluso contextos narrativos.

Vamos a hacer un análisis comparativo sobre los resultados de estas arquitecturas de colorización de imágenes, para ello hemos utilizado como base el dataset \textbf{STL-10}, bastante utilizado para clasificación con imágenes de animales y vehículos, y además un dataset complementario de flores con descripciones textuales para el enfoque guiado por texto. También probaremos el efecto que puede tener aplicar a las imágenes \textbf{segmentación previa a la colorización}, para comprobar si puede mejorar la coherencia de los resultados al proporcionar información estructural sobre los objetos presentes en la escena.


