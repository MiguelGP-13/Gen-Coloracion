\documentclass[12pt]{article}
\setlength{\parindent}{0pt}
\usepackage{float}
\usepackage{graphicx}
\usepackage{openwork}

\title{Coloración de imágenes}
\author[1]{Álvaro Felipe Pérez}
\author[1]{Miguel Gómez Prieto}
\author[1]{Hugo Gómez-Caraballo López-Romero}
\author[1]{Carlos Serrano Pinós}
\affil[1]{Estudiante CDIA, Universidad Politécnica de Madrid, Madrid}
\date{}

\begin{document}

\maketitle

\begin{abstract}
La coloración automática de imágenes en escala de grises constituye un 
desafío relevante en visión por computador, al requerir la generación de 
información cromática plausible a partir de datos incompletos. En este trabajo 
se presenta un análisis comparativo de diferentes arquitecturas de aprendizaje 
profundo aplicadas a la coloración de imágenes, incluyendo Autoencoders (AE), 
Variational Autoencoders (VAE), redes UNet, Generative Adversarial Networks (GAN), 
modelos de difusión y enfoques guiados por texto. Los experimentos se realizaron 
sobre el conjunto de datos STL‑10 y un dataset complementario de flores, con el 
objetivo de evaluar la capacidad de cada modelo para reconstruir color de 
manera realista y coherente con el contenido semántico. Se discuten las ventajas 
y limitaciones de cada aproximación, así como la influencia de la resolución 
y el tipo de condicionamiento en la calidad perceptual de los resultados. Los hallazgos contribuyen a comprender el potencial y las fronteras actuales de las técnicas de coloración basadas en deep learning. \\
\textbf{Keywords:} Image colorization, Autoencoder, UNet, Generative Adversarial Network, 
Diffusion models, Text‑guided coloration, Computer vision
\end{abstract}

\clearpage
\tableofcontents
\clearpage

\section{Introducción}

La colorización automática de imágenes en escala de grises ha sido un reto clásico en el campo de la visión por computador y el procesamiento digital de imágenes. Desde sus primeras aplicaciones en la restauración de fotografías históricas hasta su uso actual en generación de contenido digital, este problema plantea una dificultad fundamental: reconstruir información cromática que no está presente en los datos originales. La tarea no consiste únicamente en añadir color de manera arbitraria, sino en generar tonalidades plausibles y coherentes con la semántica de la escena, preservando al mismo tiempo la estructura y los detalles de la imagen.

Con el auge del aprendizaje profundo, la colorización ha experimentado un avance significativo. Las arquitecturas modernas permiten aprender representaciones latentes que capturan tanto la estructura como la distribución cromática de los datos, ofreciendo resultados mucho más realistas que los métodos tradicionales basados en reglas o en interpolación manual. En este contexto, diferentes enfoques han sido explorados: desde \textbf{Autoencoders (AE)} y \textbf{Variational Autoencoders (VAE)}, que aprenden a comprimir y reconstruir imágenes en espacios latentes, hasta \textbf{redes UNet}, que aprovechan conexiones de salto para preservar detalles espaciales durante la reconstrucción.

Por otro lado, las \textbf{Generative Adversarial Networks (GAN)} han demostrado un gran potencial en la generación de imágenes fotorrealistas, gracias a la dinámica competitiva entre generador y discriminador. En colorización, las GAN permiten producir colores más vivos y naturales, aunque a menudo requieren un entrenamiento cuidadoso para evitar inestabilidades. Más recientemente, los \textbf{modelos de difusión} han emergido como una alternativa poderosa, capaces de generar imágenes de alta calidad mediante un proceso iterativo de refinamiento desde ruido hacia datos estructurados. Estos modelos, al operar en espacios latentes comprimidos por un VAE, han mostrado resultados sobresalientes en tareas de síntesis y edición de imágenes, incluyendo la colorización.

Un aspecto adicional que ha cobrado relevancia es el \textbf{condicionamiento mediante texto}. La posibilidad de guiar el proceso de colorización con descripciones lingüísticas abre la puerta a aplicaciones creativas y personalizadas, donde el usuario puede especificar estilos, paletas cromáticas o incluso contextos narrativos. Aunque este enfoque introduce complejidad adicional, también amplía el rango de control sobre el resultado final.

En este trabajo se presenta un análisis comparativo de estas arquitecturas aplicadas a la colorización de imágenes, utilizando como base el conjunto de datos \textbf{STL-10}, ampliamente empleado en tareas de clasificación y representación visual, y un dataset complementario de flores, que aporta diversidad cromática y riqueza semántica. Además, se explora la hipótesis de que una \textbf{segmentación previa a la colorización} puede mejorar la coherencia de los resultados, al proporcionar información estructural más precisa sobre los objetos presentes en la escena.

La introducción de este estudio busca situar la colorización automática dentro del panorama actual de la inteligencia artificial aplicada a la visión por computador, destacando tanto su relevancia práctica como su interés científico. A través de la comparación de AE, VAE, UNet, GAN, modelos de difusión y enfoques guiados por texto, junto con la incorporación de segmentación como paso previo, se pretende ofrecer una visión integral de las técnicas más representativas y de su impacto en la calidad perceptual de las imágenes generadas.

\section{Estudio del Estado del Arte}
Los avances recientes en colorización automática de imágenes se estructuran en diversas familias metodológicas que definen el estado del arte en esta área. En primer lugar, los métodos basados en redes convolucionales continúan siendo un pilar fundamental dentro del aprendizaje profundo. Estas aproximaciones suelen apoyarse en variantes refinadas de arquitecturas tipo U-Net, donde se integran convoluciones dilatadas, técnicas de normalización y diversas modificaciones estructurales destinadas a mejorar la preservación del detalle y la coherencia espacial.

En paralelo, los modelos sustentados en mecanismos de atención y arquitecturas Transformer han experimentado un crecimiento notable, impulsados por su capacidad para modelar dependencias de largo alcance entre regiones distantes de la imagen. Esta propiedad se traduce habitualmente en una mayor consistencia cromática y semántica, especialmente en escenas complejas donde la comprensión global del contexto resulta determinante.

Otra línea destacada la constituyen los métodos basados en redes GAN, junto con las variantes híbridas que combinan GANs y módulos de atención. Estos sistemas han demostrado una capacidad superior para producir colorizaciones naturales y vívidas, mitigando problemas habituales como la desaturación cromática.

Asimismo, las arquitecturas híbridas con \textit{dual-decoder} representan una contribución relevante al buscar un equilibrio óptimo entre fidelidad local y coherencia global. Este tipo de diseños permite reducir artefactos característicos del proceso de colorización, como el \textit{color bleeding}, al separar explícitamente la reconstrucción estructural del razonamiento semántico.

De forma complementaria, los estudios de carácter \textit{survey} y los trabajos de \textit{benchmarking} publicados en los últimos años proporcionan una visión integradora del campo. Estos análisis comparativos revisan las limitaciones de los conjuntos de datos disponibles, evidencian la presencia de sesgos cromáticos y discuten la necesidad de métricas más representativas para evaluar adecuadamente la naturalidad, la coherencia semántica y la diversidad cromática de los modelos actuales.

Entre los modelos contemporáneos más representativos destaca DDColor, considerado uno de los enfoques con mejor equilibrio entre estructura y semántica gracias a su arquitectura basada en dos decodificadores complementarios:

\begin{itemize}
\item Pixel Decoder (Estructural): Responsable de reconstruir la resolución espacial, las texturas y la definición de bordes.
\item Color Decoder (Semántico): Basado en tecnología Transformer y \textit{Color Queries}, permitiendo asignar colores coherentes y vibrantes en función del contenido semántico, sin comprometer los límites estructurales.
\end{itemize}

La combinación de ambas ramas permite obtener colorizaciones altamente detalladas y cromáticamente realistas, superando a numerosos métodos previos basados exclusivamente en GANs \cite{Kang_2023_ICCV}.

Si el objetivo es maximizar el realismo visual, uno de los modelos más avanzados es CtrlColor, basado en técnicas de difusión. Este método mejora significativamente las limitaciones presentes en trabajos previos, especialmente en relación con el \textit{color bleeding}. Su principal aportación consiste en un marco unificado que integra tres modos de operación dentro de una única red, sin necesidad de reentrenamiento:

\begin{itemize}
\item Incondicional: Permite la colorización automática al estilo de DDColor.
\item Guiado por texto: Utiliza prompts para orientar la paleta cromática.
\item Guiado por trazos: El usuario puede especificar trazos de color, que el modelo propaga respetando la estructura geométrica de la imagen.
\end{itemize}

Esta versatilidad es posible gracias a una arquitectura de difusión latente que incorpora un Codificador de Contenido especializado. Dicho codificador actúa como una restricción estructural explícita, garantizando que los bordes y formas originales limiten el proceso generativo y evitando que la difusión genere colores fuera de las regiones correspondientes \cite{liang2024control}.

Finalmente, la vanguardia actual apunta hacia modelos basados en Flow Matching, entre los que destaca ModFlows. Este enfoque marca una transición conceptual desde los métodos de difusión hacia los denominados \textit{flujos rectificados}. Su principal innovación radica en abandonar el proceso estocástico y progresivo de eliminación de ruido característico de la difusión, sustituyéndolo por una trayectoria determinista que transporta directamente la distribución de color desde una imagen de referencia hasta la imagen objetivo. Este cambio permite acelerar drásticamente la inferencia, entre 20 y 30 veces respecto a los modelos de difusión, y garantiza resultados más estables, coherentes y libres de alucinaciones \cite{Larchenko_2025_ModFlows}.
\section{Autoencoder}
Comenzaremos nuestro proyecto probando ciertos modelos sencillos para ver su desempeño sobre el dataset STL-10 \cite{coates2011analysis}. En este apartado hablaremos sobre dos diferentes tipos de autoencoders cuya única entrada es una imagen en blanco y negro y su salida es la imagen reconstruida a color. 

\subsection{U-Net Autoencoder}
El modelo implementado sigue una arquitectura tipo U-Net simétrica \cite{ronneberger2015unet}. A diferencia de un autoencoder secuencial estándar, esta red incorpora conexiones de salto (skip connections) que unen las capas del codificador con sus correspondientes en el decodificador.
La arquitectura se compone de tres bloques funcionales detallados a continuación: 

\begin{itemize}
    \item Encoder: recibe un tensor de dimensiones (96×96×1). Consta de 3 bloques secuenciales. Cada bloque aplica una convolución de 3×3 con activación ReLU y padding "same" para mantener las dimensiones, seguida inmediatamente de una operación de Max Pooling que reduce el tamaño espacial a la mitad.
    \item Cuello de botella:  comprime al máximo la información (12x12). Se utiliza una capa convolucional de 256 filtros para conectar ambos bloques.
    \item Decoder: aplicamos UpSampling2D para duplicar el tamaño de la imagen cada vez. Uno de sus detalles más importantes es que se concatena información con capas del encoder permitiendo al modelo recordar la ubicación exacta de los bordes y texturas que suelen perderse durante la compresión del Max Pooling.\\
    Finalmente, se utiliza una capa convolucional con activación sigmoide que proyecta el resultado a los tres canales R, G, B.
\end{itemize}

\subsection{Residual U-Net Autoencoder}
Este seugndo modelo propuesto sustituye las convoluciones estándar por Bloques Residuales e introduce mecanismos de normalización y submuestreo aprendible \cite{zhang2018road}. A continuación se detallan sus componentes clave:

\begin{itemize}
    \item Bloques residuales: Esta es la unidad fundamental de la red. A diferencia de una capa convolucional simple que intenta aprender una función H(x) directamente, este bloque intenta aprender el residuo F(x), de tal forma que la salida sea H(x)=F(x)+x.
    \item Encoder: similar al anterior
    \item Decoder: se diferencia del anterior porque utiliza capas deconvolucionales en lugar de capas upsampling. Las capas deconvolucionales tienen pesos entrenables. La red aprende la mejor manera de aumentar la resolución de la imagen, recuperando texturas y bordes de forma más inteligente que una simple interpolación bilineal.\\
    Finalmente, se utiliza una capa convolucional con activación sigmoide que proyecta el resultado a los tres canales R, G, B.
\end{itemize}

\subsection{Justificación de modelos}
Se descarta el uso de autoencoders secuenciales simples debido a que el proceso de compresión provoca una pérdida irreversible de información espacial. Esto provoca el fenómeno de color bleeding, donde el color se desborda de los contornos al no existir una referencia clara de los bordes durante la reconstrucción.\\

Por el contrario,  la elección de las arquitecturas U-Net y ResU-Net se fundamenta en su capacidad para resolver este compromiso:
\begin{itemize}
\item Recuperación Espacial: Las skip connections reintroducen los detalles de alta resolución bordes y texturas directamente desde la entrada, garantizando que el color se aplique con cierta precisión sobre la estructura original.

\item Capacidad de Aprendizaje: La inclusión de bloques residuales permite aumentar la profundidad de la red sin degradar el entrenamiento, facultando al modelo para aprender relaciones de color complejas y sutiles que un modelo superficial no podría captar.
\end{itemize}

\subsection{Entrenamiento}
Ambos modelos se compilan utilizando el optimizador Adam y la función de pérdida MSE, estándar para regresión de píxeles. Sin embargo, para la ResU-Net se ha reducido el learning rate a 0.0005 para garantizar la estabilidad del gradiente dada su mayor profundidad.\\
Además, es interesante comentar la diferencia significativa en la carga computacional debido a la complejidad de las arquitecturas:
\begin{itemize}
\item Modelo U-Net: menos de 2 min/epoch. Al ser más ligero, permite un entrenamiento más extenso (40 epochs).

\item Modelo ResU-Net: sobre 8 min/epoch. La inclusión de bloques residuales, normalización por lotes y mayor número de filtros (hasta 512) incrementa drásticamente las operaciones por ciclo, lo que ha obligado a limitar el entrenamiento.
\end{itemize}

Para acelerar el entrenamiento de ambos modelos se ha implementado una política de Mixed Precision (FP16).

\subsection{Resultados}
Mostramos en la siguiente tabla las métricas obtenidas en test de ambos modelos:

\begin{table}[H]
\centering
\caption{Comparación de métricas entre U-Net y ResU-Net}
\vspace{0.2cm}
\begin{tabular}{lcc}
\hline
\textbf{Métrica} & \textbf{U-Net} & \textbf{ResU-Net} \\ \hline
SSIM  & 0.9323 & 0.9360 \\ 
PSNR  & 25.3312 & 25.0441 \\ 
LPIPS & 0.1345 & 0.1466 \\ \hline
\end{tabular}
\end{table}
Los resultados cuantitativos muestran un rendimiento muy parejo entre ambos modelos, con ligeras variaciones atribuibles a la diferencia en el tiempo de entrenamiento (40 épocas vs 5 épocas):

\begin{itemize}
    \item \textbf{SSIM (estructura):} La ResU-Net (0.9360) supera ligeramente a la U-Net. Esto confirma que los bloques residuales y la mayor capacidad de la red ayudan a reconstruir mejor la estructura y los bordes de la imagen, incluso con menos entrenamiento.

    \item \textbf{PSNR y LPIPS (fidelidad y percepción):} La U-Net obtiene mejores puntuaciones (mayor PSNR, menor LPIPS). Esto se debe a que, al haber entrenado durante 40 épocas, ha tenido tiempo de converger hacia una solución más estable y minimizar el error medio, mientras que la ResU-Net (con solo 5 épocas) quizá aún no haya alcanzado su potencial máximo.
\end{itemize}

Observamos a continuación ciertas imágenes:

\begin{figure}[H]
    \centering
    \includegraphics[width=0.5\linewidth]{images/unet.png}
    \caption{Imágenes generadas con U-Net}
    \label{fig:unet}
\end{figure}

\begin{figure}[H]
    \centering
    \includegraphics[width=0.5\linewidth]{images/resunet.png}
    \caption{Imágenes generadas con ResU-Net}
    \label{fig:resunet}
\end{figure}

Vemos una mayor vivacidad y acierto con los colores de la U-Net que corresponde con los mejores resultados obtenidos en PSNR y LPIPS, pero vemos que su estructura es algo peor porque sus bordes coloreados son algo desacertados.

\subsection{Conclusiones}
El análisis comparativo entre las dos arquitecturas propuestas revela un claro compromiso entre complejidad y rendimiento.\\
Si bien la ResU-Net demostró una superioridad estructural latente (mejor SSIM) gracias a sus bloques residuales, su elevado coste computacional limitó la convergencia cromática en comparación con la U-Net, que logró mayor fidelidad de color gracias a un entrenamiento más extenso. \\
A pesar de las limitaciones intrínsecas del modelo autoencoder y la pérdida MSE, que tienden a generar tonos desaturados ante la incertidumbre, los resultados cualitativos han superado las expectativas iniciales para modelos de esta complejidad,

\section{VAE}
En este apartado, probaremos dos diferentes VAE sobre el mismo dataset y cuya entrada y salida es la misma que los autoencoders anteriores.
\subsection{VAE Denso}
Este primer modelo implementa un VAE clásico con un cuello de botella denso \cite{kingma2013auto}:
\begin{itemize}

\item Encoder: Rompe la estructura espacial de la imagen. Tras las convoluciones, utiliza Flatten para convertir el mapa de características 2D en un vector plano 1D, comprimiéndolo todo mediante capas Dense.

\item Latent Space: Es un vector de números, sin altura ni anchura.

\item Decoder: Debe "reconstruir" el espacio desde cero. Utiliza una capa densa para expandir el vector y luego Reshape para forzarlo a tener forma de imagen antes de aplicar las deconvoluciones.
\end{itemize}

\subsection{VAE Convolucional}
Este modelo implementa un Autoencoder Variacional que mantiene la topología 2D de la imagen a lo largo de todo el proceso, evitando las capas densas que destruyen la información espacial. Sus componentes técnicos clave son: 

\begin{itemize}

\item Entrenamiento Estocástico: Implementa el Truco de Reparametrización 
\[
z = \mu + \sigma \cdot \varepsilon
\]
separando la aleatoriedad para permitir el cálculo de gradientes y el entrenamiento de la red.

\item Encoder Aprendible: Sustituye el MaxPooling fijo por Convoluciones con Stride, otorgando a la red la capacidad de aprender cómo reducir la dimensión sin perder información estructural crítica. Utiliza activación LeakyReLU para garantizar la estabilidad del flujo de gradientes.

\item Reconstrucción: El decodificador recupera la resolución original mediante Convoluciones Transpuestas, proyectando los detalles de color directamente desde los mapas de características espaciales del cuello de botella.

\end{itemize}
\subsection{Justificación de modelos}
Estos modelos se implementan para abordar la ambigüedad intrínseca de la colorización porque un objeto gris puede tener múltiples colores válidos.

Mientras que los modelos deterministas tienden a promediar colores generando tonos apagados ante la duda, los VAEs aprenden una distribución de probabilidad, permitiendo generar colores más vivos y diversos.

Además, la creación del modelo VAE convolucional se justifica porque al mantener un espacio latente tensorial no aplanado, preserva la correlación espacial de los píxeles, resultando en una reconstrucción geométrica superior.

\subsection{Entrenamiento}
Ambos modelos se basan en dos términos fundamentales en su función de pérdida: un término de reconstrucción y un término de regularización mediante divergencia KL. Sin embargo, en cada modelo lo implementamos de diferente forma

El primer modelo, utiliza MSE para medir la discrepancia entre la imagen reconstruida y la imagen real en color. Tras observar los resultados de este modelo, decidimos implementar MAE en el segundo modelo porque penaliza de forma lineal los errores, lo que tiende a producir reconstrucciones más nítidas.\\
Por otra parte, ambos modelos emplean la misma expresión para la divergencia KL. Sin embargo, en el primer modelo la KL se añade directamente a la pérdida total porque es un espacio latente relativamente pequeño.
\[
\mathcal{L} = \mathcal{L}_{\text{rec}} + \mathcal{L}_{\text{KL}}
\]
Esto ayuda a que la distribución latente sea estable y bien organizada.

En el segundo modelo la KL se multiplica por un peso \(\lambda\) muy pequeño.
\[
\mathcal{L} = \mathcal{L}_{\text{rec}} + \lambda \cdot \mathcal{L}_{\text{KL}}
\]
La razón es que el espacio latente ahora es mucho más grande. Si se utilizara la KL sin ponderación, esta dominaría por completo la pérdida, impidiendo que el modelo aprenda a reconstruir correctamente.

El tiempo de entrenamiento de ambos modelos es muy similar (2 min/epoch). Para estos modelos se ha desactivado Mixed Precision por sencillez.

\subsection{Resultados}
Mostramos en la siguiente tabla las métricas obtenidas en test de ambos modelos:

\begin{table}[H]
\centering
\caption{Comparación de métricas entre VAE Denso y VAE Convolucional}
\vspace{0.2cm}
\begin{tabular}{lcc}
\hline
\textbf{Métrica} & \textbf{VAE Denso} & \textbf{VAE Convolucional} \\ \hline
SSIM  & 0.0600 & 0.9338 \\ 
PSNR  & 6.3406 & 25.1147 \\ 
LPIPS & 0.6884 & 0.1485 \\ \hline
\end{tabular}
\end{table}

La comparación cuantitativa evidencia una diferencia abismal en el rendimiento de ambas arquitecturas, validando la importancia crítica de la preservación espacial en el espacio latente:

Los valores obtenidos (SSIM: 0.06, PSNR: 6.34) indican un colapso estructural casi total del VAE Denso. Al aplanar los mapas de características y utilizar capas densas, el modelo pierde la correlación espacial de los píxeles, resultando en una salida que no logra reconstruir ni siquiera la forma de los objetos, comportándose prácticamente como ruido aleatorio.

Por el contrario, el modelo totalmente convolucional alcanza un SSIM de 0.9338, situándose al mismo nivel de calidad que los modelos autoencoder. Esto demuestra que mantener la topología 2D en el cuello de botella es un requisito indispensable para la colorización.

La métrica perceptual confirma lo anterior (LPIPS): el VAE Denso (0.6884) genera imágenes irreconocibles para el ojo humano, mientras que el CVAE (0.1485) produce resultados coherentes y naturales.

Con una inspección visual observamos reflejados los resultados obtenidos cuantitativamente por las métricas:

\begin{figure}[H]
    \centering
    \includegraphics[width=0.5\linewidth]{images/dvae.png}
    \caption{Imágenes generadas con VAE Denso}
    \label{fig:placeholder}
\end{figure}

\begin{figure}[H]
    \centering
    \includegraphics[width=0.5\linewidth]{images/cvae.png}
    \caption{Imágenes generadas con VAE Convolucional}
    \label{fig:placeholder}
\end{figure}

Observamos una imagen muy borrosa del primer VAE indicando que no ha sido capaz de reconstruir la imagen. Con el segundo VAE sí tenemos una reconstrucción óptima de la imagen pero la colorización no es correcta.

\subsection{Conclusiones}
La experimentación con modelos variacionales evidencia que la preservación de la topología en el espacio latente es un requisito indispensable, habiendo logrado el VAE convolucional rescatar la coherencia estructural tras el colapso del enfoque denso. Sin embargo, pese a esta validación arquitectónica, es necesario señalar que los resultados visuales finales no logran superar la calidad obtenida por los autoencoders. La restricción de regularización impuesta por la Divergencia KL, sumada a la optimización de pérdidas píxel a píxel, tiende a producir imágenes con texturas suavizadas y menor definición cromática. Esta limitación perceptiva justifica el siguiente paso en la investigación: la transición hacia las Redes GAN, cuyo mecanismo de discriminación busca romper precisamente esta barrera de borrosidad para forzar la generación de detalles de alta frecuencia y un realismo visual superior.
\section{GAN}

Tras evaluar los modelos Autoencoders y VAEs, exploramos ahora una 
arquitectura GAN para abordar la colorización desde este enfoque. 
A diferencia de los modelos anteriores, cuyo objetivo es minimizar un error de reconstrucción, 
las GAN introducen el discriminador que fuerza al generador a producir imágenes más realistas. Esto permite obtener resultados más vivos y definidos, 
especialmente en regiones donde la decisión del color es ambigua.

Nuestro modelo sigue una estructura similar al método \textit{pix2pix}, donde el objetivo es 
aprender una transformación de imágenes en escala de grises (canal L del espacio Lab) a 
imágenes a color (canales a y b). La tarea se formula como un problema de 
\textit{image-to-image translation}.

\subsection{Arquitectura del Generador (U-Net)}
Para el generador empleamos una U-Net, ya que sus conexiones de salto permiten mantener 
la estructura espacial con gran fidelidad. En el contexto de la colorización, esto es 
especialmente importante: aunque el color puede ser ambiguo, la ubicación de los bordes y 
texturas debe conservarse con precisión.

El generador recibe como entrada un tensor $(128\times128\times1)$ correspondiente al 
canal de luminancia L normalizado. Su estructura se divide en:

\begin{itemize}
    \item \textbf{Encoder:} consiste en cinco bloques de \textit{downsampling}, implementados con 
    convoluciones de 4×4 y stride 2. Cada bloque duplica el número de filtros (64, 128, 256, 512, 512),
    reduciendo la imagen hasta un mapa de 4×4. Se utiliza LeakyReLU como activación y 
    BatchNorm en todos los niveles excepto el primero.

    \item \textbf{Decoder:} está formado por cuatro bloques de \textit{upsampling} mediante 
    \textit{Conv2DTranspose}. Los dos primeros incluyen \textit{dropout} para introducir ruido 
    estructurado y evitar que el generador memorice patrones del conjunto de entrenamiento.
    Cada capa del decoder concatena su salida con la activación correspondiente del encoder 
    (skip connection), restaurando así la información espacial perdida.

    \item \textbf{Salida:} una convolución transpuesta final de 4×4 produce una imagen 
    $(128\times128\times2)$ con activación \textit{tanh}, que representa los canales a y b ya 
    normalizados.
\end{itemize}

Esta arquitectura permite que el modelo coloree respetando forma, contornos y textura, incluso 
en regiones donde un autoencoder simple tiende a producir manchas o difuminados.

\subsection{Arquitectura del Discriminador (PatchGAN)}
El discriminador evalúa si una imagen generada es realista comparándola con la 
imagen real. En lugar de clasificar la imagen completa, utilizamos un discriminador 
PatchGAN, que divide la imagen en parches locales y decide para cada uno si el contenido 
luminancia/color es coherente.

Su estructura consiste en:

\begin{itemize}
    \item Concatenación de la entrada: el discriminador recibe una pareja formada por la 
    imagen en escala de grises (L) y la imagen color (real o generada), que se combinan en 
    un tensor $(128\times128\times3)$.

    \item Tres bloques de \textit{downsampling} con convoluciones de 4×4, stride 2 y LeakyReLU, 
    que reducen progresivamente la resolución y capturan texturas locales.

    \item Un nivel adicional de convolución sin \textit{downsampling}, que amplía el campo 
    receptivo sin perder detalle.

    \item Una última convolución produce un mapa de activación $(14\times14\times1)$ donde 
    cada valor indica si el parche correspondiente parece real o generado.
\end{itemize}

PatchGAN es especialmente útil porque penaliza los errores de coloración finos (bordes, 
texturas, transiciones) sin imponer restricciones globales demasiado rígidas.

\subsection{Función de pérdida}
La pérdida total combina dos términos:

\begin{itemize}
    \item \textbf{Pérdida adversaria} (\( \mathcal{L}_{GAN} \)): obliga al generador a producir imágenes 
    indistinguibles para el discriminador. Está implementada mediante binary crossentropy.

    \item \textbf{Pérdida L1} (\( \mathcal{L}_{L1} \)): calcula el MAE entre los canales de 
    color reales y generados. Esto estabiliza el entrenamiento y evita colores 
    fuera de rango o inconsistentes.
\end{itemize}

La pérdida global se define como:
\[
\mathcal{L}_G = \mathcal{L}_{GAN} + \lambda \cdot \mathcal{L}_{L1},
\]
donde utilizamos \(\lambda = 100\).

\subsection{Entrenamiento}
Tanto el generador como el discriminador se entrenan con el optimizador Adam 
(\texttt{lr = 2e-4}, \(\beta_1 = 0.5\)), como se suele recomendar para que el entrenamiento de la gan sea estable.

Durante cada iteración:

\begin{enumerate}
    \item El generador predice los canales a y b a partir del canal L.
    \item El discriminador recibe tanto la imagen real (L + ab) como la generada.
    \item Se calcula la pérdida, la pérdida L1 y los gradientes de ambos modelos.
    \item Se actualizan las dos redes de forma independiente.
\end{enumerate}

Entrenamos el modelo durante 40 épocas utilizando un subconjunto del dataset STL-10 
(20.000 imágenes no etiquetadas para entrenamiento, 1.000 para validación y 3.000 para test). 
Todas las imágenes se transforman al espacio Lab mediante \textit{skimage} y se normalizan a 
valores entre \([-1, 1]\). El tamaño final utilizado es \(128\times128\), lo que equilibra detalle y 
tiempo de cómputo.

\subsection{Resultados}
En las primeras épocas las colorizaciones muestran tonos desaturados y cierta 
inconsistencia en regiones complejas. A medida que avanza el 
entrenamiento, el modelo aprende a producir colores más vivos y transiciones suaves.

Frente a los autoencoders, la GAN genera imágenes perceptualmente más atractivas, con 
mayor viveza y contraste. Sin embargo, también muestra cierta variabilidad en 
zonas ambiguas, consecuencia directa de la optimización adversaria y del carácter 
condicional del modelo.
\section{Colorización con modelos de difussion}
\input{Capítulos/text-guided}
\section{Conclusión}
En esta sección se presentan las conclusiones generales del trabajo, apoyadas en las métricas
cuantitativas y cualitativas obtenidas para cada modelo de colorización. Los resultados permiten
comparar de manera objetiva el rendimiento de arquitecturas basadas en autoencoders, variational
autoencoders, redes generativas adversarias y modelos de difusión, destacando sus fortalezas y
limitaciones en términos de estructura, fidelidad numérica y percepción visual.

\begin{table}[htbp]
    \centering
    \caption{Comparación de métricas entre modelos de colorización}
    \label{tab:comparacion_modelos}
    \begin{tabular}{lcccc}
        \hline
        \textbf{Modelo}       & \textbf{SSIM} $\uparrow$ & \textbf{LPIPS} $\downarrow$ & \textbf{MSE} $\downarrow$ & \textbf{CIEDE2000} $\downarrow$ \\ \hline
        U-Net (AE)            & 0.9323                   & 0.1345                      & \textbf{190.6}            & --                              \\
        ResU-Net (AE)         & \textbf{0.9360}          & 0.1466                      & 203.7                     & --                              \\ 
        VAE Denso             & 0.0600                   & 0.6884                      & 15103.6                   & --                              \\ 
        VAE Convolucional     & 0.9338                   & 0.1485                      & 200.5                     & --                              \\ 
        GAN clásica           & 0.9097                   & 0.1628                      & 278.7                     & --                              \\ 
        GAN + Segmentación    & 0.9073                   & 0.1660                      & 297.9                     & --                              \\ 
        Diffusion UNet        & 0.9152                   & 0.1437                      & 461.3                     & 11.86                           \\
        Diffusion Fine-tuning & 0.8636                   & 0.1416                      & 554.0                     & 13.76                           \\ 
        LoRA Stable Diffusion & 0.9034                   & \textbf{0.1322}             & 499.1                     & \textbf{11.77}                  \\ 
    \end{tabular}
\end{table}
\FloatBarrier

A partir de la comparación mostrada en la Tabla~\ref{tab:comparacion_modelos}, se observa que los
autoencoders (U-Net y ResU-Net) ofrecen la mejor preservación estructural y menor error numérico,
mientras que las GANs aportan mayor realismo visual y saturación cromática. Finalmente, los modelos
de difusión, especialmente LoRA sobre Stable Diffusion, destacan en métricas perceptuales como LPIPS
y CIEDE2000, lo que evidencia su potencial para aproximarse más a la percepción humana del color.
En conjunto, los resultados confirman que cada enfoque aporta ventajas específicas y que la
integración de técnicas multimodales constituye una línea prometedora para futuras investigaciones.



\subsection{Colorización guiada por texto}

La colorización guiada por texto ha demostrado ofrecer resultados muy sólidos cuando se dispone de imágenes acompañadas de descripciones ricas y precisas, como ocurre en el caso de las flores. Consideramos que esta estrategia podría extenderse a otros dominios donde la colorización sea relevante; sin embargo, la principal limitación reside en la disponibilidad de \textit{captions} de calidad, algo poco común en la mayoría de los datasets existentes.
Pensamos que sería factible aproximarse a un colorizador más general para todos los usos si se adoptara un enfoque de entrenamiento similar al de modelos como CLIP, que aprovechan casi todas las imágenes que hay en Internet junto a su texto alternativo. No obstante, un esfuerzo de esta magnitud requeriría modelos de mayor capacidad y recursos computacionales sustancialmente superiores.

\section*{Agradecimientos}
...

\printbibliography

\end{document}